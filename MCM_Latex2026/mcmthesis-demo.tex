%%
%% This is file `mcmthesis-demo.tex',
%% generated with the docstrip utility.
%%
%% The original source files were:
%%
%% mcmthesis.dtx  (with options: `demo')
%% 
%% -----------------------------------
%% This is a generated file.
%% 
%% Copyright (C) 2010 -- 2015 by latexstudio
%%       2014 -- 2019 by Liam Huang
%%       2019 -- present by latexstudio.net
%% 
%% This work may be distributed and/or modified under the
%% conditions of the LaTeX Project Public License, either version 1.3
%% of this license or (at your option) any later version.
%% The latest version of this license is in
%%   http://www.latex-project.org/lppl.txt
%% and version 1.3 or later is part of all distributions of LaTeX
%% version 2005/12/01 or later.
%% 
%% The Current Maintainer of this work is latexstudio.net.
%% 
%% !Mode:: "TeX:UTF-8"
\documentclass{mcmthesis}
 %\documentclass[CTeX = true]{mcmthesis}  % 当使用 CTeX 套装时请注释上一行使用该行的设置
\mcmsetup{tstyle = \color{red}\bfseries,  % 修改题号,队号的颜色和加粗显示,黑色可以修改为 black
          tcn = 2615744, problem = C,     % 修改队号,参赛题号
          sheet = true, titleinsheet = true, keywordsinsheet = true,
          titlepage = false, abstract = true}

  %四款字体可以选择
  %\usepackage{times}
  %\usepackage{newtxtext}
  %\usepackage{palatino}
\usepackage{txfonts}

% 允许从项目根目录(main.tex)编译时也能找到图片
\graphicspath{{figures/}{MCM_Latex2026/figures/}}

\usepackage{caption}  
\captionsetup[figure]{font=small}   % 将图标题字号设为small
\captionsetup[table]{font=small}    % 将图标题字号设为small

\usepackage{indentfirst}  % 首行缩进,注释掉,首行就不再缩进
\usepackage{esint}        % 积分号
\usepackage{subcaption}   % 并排图片
\usepackage{floatrow}     % 并排图片

\usepackage{float}
\floatstyle{plaintop}
\restylefloat{table}

\usepackage{xcolor}       % 颜色支持
\usepackage{tabularray}   % 增强表格功能

\usepackage[style=numeric,backend=biber]{biblatex}  % APA风格: style=apa
\ExecuteBibliographyOptions{sorting=none}           % 按引用顺序
% 兼容两种编译方式:
% 1) 在 MCM_Latex2026 目录内编译:references.bib
% 2) 在项目根目录编译:MCM_Latex2026/references.bib
\IfFileExists{references.bib}{\addbibresource{references.bib}}{}
\IfFileExists{MCM_Latex2026/references.bib}{\addbibresource{MCM_Latex2026/references.bib}}{}

% 修复fancyhdr警告
\setlength{\headheight}{14pt}
\addtolength{\topmargin}{-1.6pt}  % 可选:保持总版心不变

\setcounter{tocdepth}{2}  % 设置目录到二级标题, 将目录控制在1页
                          % 若可接受目录2页, 注释掉该句, 将到三级标题
                          
\title{Veils of Uncertainty: Weaving Risk into the Tapestry of Preservation
Under the Weather's Watch}

% \author{\small \href{https://www.latexstudio.net/}
%   {\includegraphics[width=7cm]{mcmthesis-logo}}}

\date{\today}

\begin{document}

\begin{abstract}
\par As the tapestry of nature weaves its unpredictable patterns, humanity’s quest for stability
becomes ever more pressing. In the shadow of uncertainty, we find resilience, crafting shields
against the tempests of fate.

First, we establish a Risk Analysis model to comprehensively assess the Expected Annual
Loss (EAL) from extreme weather in terms of population, building, and agriculture. The
assessment for each aspect is calculated from three perspectives: natural hazard exposure, Historic
Loss Ratio, and the likelihood risk factor of natural hazard annualized frequency. Community
Risk Factor (CRF) is calculated from social vulnerability and community resilience.
EAL and CRF are used to quantify the risk levels of various regions and rank them using the
K-means algorithm, resulting in a risk level map of the United States.
Second, we develop a Risk-incorporated Capital Asset Pricing Model (CAPM) to aid
insurance companies in underwriting decisions. This model combines market return rates, the
risk-free rate, and bankruptcy theory with a $10\%$ bankruptcy probability to set insurance rates.
It evaluates if the region's residents can afford these premiums, providing decision-making advice
for insurance companies.

More specifically, we apply our Risk-incorporated Capital Asset Pricing Model in Los
Angeles and Gorontalo. In Los Angeles, insurance companies see high profits and low risks.
However, in Gorontalo, the required premium for $\$10,000$ coverage is $\$342.745$, beyond local
affordability. We recommend insurance securitization and partnerships with local governments
to reduce premiums. Consequently, Gorontalo residents could pay just $\$137.25$ annually,
with companies projecting $\$245$ million in revenue.

Third, we establish a Building Preservation Model, selecting seven secondary indicators
such as the annual number of visitors and construction cost, and three primary indicators: cultural
values and community influence, economy, and history. These are weighted using the
Sperman-CRITIC algorithm and AHP method to calculate building value, combined with
risk levels to determine the preservation level of buildings. Based on the preservation level, the
community's investment and measures for building protection can be determined.
Then our models inform investment and protection strategies for Tokyo Tower, acknowledging
its value and the necessity of preservation in an earthquake zone. We communicate these
findings and propose protection measures to the Tokyo Tower community.

Finally, we analyze the sensitivity and robustness of our models, the models can change
the insurance rate sensitively according to the change of the market predicted return and the
slight error of the risk factor calculation will not affect the models’ result, which verifies the
sensitivity and robustness of our models. In addition we analyze the strengths and weaknesses
of the models.

\begin{keywords}
Risk Analysis, Risk-Capital Asset Pricing Model, Sperman-CRITIC, AHP,
Building Preservation Model
\end{keywords}

\end{abstract}
\maketitle

%% Generate the Table of Contents, if it's needed.
\tableofcontents

\newpage

\section{Introduction}
\subsection{Problem Background}

Competitive reality shows typically operate through two evaluation channels: (i) judges’ scores that reflect professional assessment of performance quality, and (ii) audience votes that represent public preference.
\noindent \emph{Dancing with the Stars} is a canonical example where the show has repeatedly adjusted how these two channels are combined and---after major controversies---introduced institutional corrections such as the ``Bottom Two + Judges' Save'' mechanism to balance meritocracy with participatory entertainment.

A core complication is that while judges' scores, eliminations, and final placements are observable, the public voting counts/shares are typically undisclosed; media explanations suggest that withholding vote details is meant to avoid biasing subsequent voting behavior. This turns rule evaluation into an inverse inference problem with a missing key variable: to assess whether a rule is fairer or more exciting, we must first reconstruct the latent voting process consistent with observed outcomes.

\begin{figure}[H]    % H表示强制固定在当前位置
\small
\centering
\includegraphics[width=0.8\textwidth]{fig1.png}
\caption{Our work} \label{fig1}
\end{figure}

\subsection{Restatement of the Problem}

Given the season-by-week panel data provided in the problem, we structure the study into a coherent four-task pipeline:

\begin{enumerate}
\item \textbf{Task 1 (Inverse inference of hidden votes):} Infer weekly audience vote shares (or equivalent votes) using only observable judges' scores and elimination/placement outcomes, and assess consistency with the observed eliminations and final placements.
\item \textbf{Task 2 (Counterfactual rule evaluation):} Assuming audience preferences remain unchanged, build a counterfactual simulation to compare alternative merging rules (e.g., percentage-based, rank-based, and variants with Judges' Save) in terms of elimination paths, winners, and systematic bias.
\item \textbf{Task 3 (Mechanism attribution---pro dancer \& celebrity traits):} Using the inferred votes from Task~1, quantify how professional partners and celebrity attributes differentially affect the judges' channel versus the audience channel.
\item \textbf{Task 4 (Designing a better rule):} Propose a new weekly merging/elimination system that is more ``fair'' and/or more ``exciting,'' and justify its adoption using the same simulator and metrics.
\end{enumerate}

\subsection{Our work}

In order to clearly illustrate our work, we draw the flowchart Figure \ref{fig2}.

\begin{figure}[H]    % H表示强制固定在当前位置
\small
\centering
\includegraphics[width=0.8\textwidth]{fig2.png}
\caption{Our work} \label{fig2}
\end{figure}

\section{Assumptions and Justifications}

Considering those practical problems always contain many complex factors, first of all,
we need to make reasonable assumptions to simplify the model, and each hypothesis is closely
followed by its corresponding explanation:

\begin{itemize}
\item {\bf Assumption:} The data we use are accurate and valid.
\item {\bf Justification:} Our data is collected from the World Bank and some other official web
sites and research papers. it’s reasonable to assume the high quality of their data.
\item {\bf Assumption:} The regions under study will remain peaceful and stable, with no significant
events other than natural disasters occurring in the foreseeable future.
\item {\bf Justification:} A stable capital market environment provides a predictable framework
within which we can project our expected returns. It is important to note that this assumption
does not negate the potential impact of natural disasters.
\item {\bf Assumption:} The estimated figures for each region represent an average level of performance
or condition for that area.
\item {\bf Justification:} For the purposes of this study, treating each region as a cohesive entity
allows for a more streamlined analysis. This methodological approach simplifies the
complexity inherent in regional studies by focusing on aggregate data, thereby providing
a generalized view of each area's performance or condition.
\end{itemize}

\section{Notations}

The key mathematical notations used in this paper are listed in Table \ref{tbl1}. 

%% 三线表
\begin{table}[H]     % H表示强制固定在当前位置
\small
\centering           % 设置居中
\caption{Notations}  % 表标题
\label{tbl1}         % 设置表的引用标签
\begin{tabular}{cl}  % 2列, c表示居中对齐, l和r表示左右对齐
\toprule             % 画顶端横线
{\bf Symbol}    &  {\bf Definition}     \\
\midrule             % 画中间横线
$EAL$  & Expected Annual Loss  \\
$SV$   & Social Vulnerability  \\
$CR$   & Community Resilience  \\
$CRF$  & Community Risk Factor \\
$HLR$  & Historic Loss Ratio   \\
\bottomrule         % 画底部横线
\end{tabular}
\vspace{2pt} % 稍微调整间距
\begin{flushleft} \small
$^{*}$ There are some variables that are not listed here and will be discussed in detail in each
section.
\end{flushleft}
\end{table}

\section{Insurance Pricing and Decision Model}

\subsection{Data Collection}

%% 三线表
\begin{table}[H]     % H表示强制固定在当前位置
\small
\centering           % 设置居中
\caption{Notations}  % 表标题
\label{tbl2}         % 设置表的引用标签
\begin{tabular}{ll}  % 2列, c表示居中对齐, l和r表示左右对齐
\toprule             % 画顶端横线
{\bf Database Names}     &  {\bf Database Websites}     \\
\midrule             % 画中间横线
Los Angeles              & \url{https://geohub.lacity.org/}  \\
National Risk Index      & \url{https://www.fema.gov/}  \\
Bureau of the Census     & \url{https://data.census.gov/}  \\
South Carolina           & \url{https://www.sc.edu/}  \\
World Bank               & \url{https://data.worldbank.org/}  \\
\bottomrule         % 画底部横线
\end{tabular}
\end{table}

%% 定理类环境
% \begin{Theorem} \label{thm1}
% Let $A$ ...
% \end{Theorem}
% 
% \begin{Lemma} \label{lem1}
% ...
% \end{Lemma}
% 
% \begin{proof}
% The proof of theorem.
% \end{proof}

\subsection{Risk Analysis}

\subsubsection{Expected Annual Loss (EAL)}

The $EAL$ value are in units of dollars, representing the community’s average economic
loss from natural hazards each year. $EAL$ is calculated using a multiplicative equation that
considers the consequence risk factors of natural hazard exposure, $HLR$ (Historic Loss Ratio),
and the likelihood risk factor of natural hazard annualized frequency. The $EAL$ value for each
consequence type is calculated by multiplying the exposure value of an area by the estimated
annualized frequency and the $HLR$.

\begin{equation}
EAL = Exposure \times Annualized~Frequency \times HLR
\end{equation}

The total $EAL$ value for each hazard type is the sum of three different types of consequences:
population (P), building (B), and agriculture (A).

\begin{equation}
EAL_{hazard type} = EAL_{hazard type(P)} + EAL_{hazard type(B)}
 + EAL_{hazard type(A)}
\end{equation}

We add up the $EAL$ values for $18$ types of hazards to obtain the composite $EAL$ value.

\begin{equation}
EAL_{composite} = \sum_{i=1}^{18} EAL_{hazard type}
\end{equation}

In order to describe our process of calculating the value of EAL more clearly. We draw
the flowchart Figure \ref{fig2} below.

\begin{figure}[H]    % H表示强制固定在当前位置
\small
\centering
\includegraphics[width=0.75\textwidth]{fig2.png}
\caption{Calculation procedure for EAL} \label{fig2}
\end{figure}

A national ranking is computed for the composite $EAL$ value of each community. The
resulting values are then converted into national percentiles to produce an $EAL$ score for each
community. The $EAL$ scores are very helpful for the risk calculation in the following text.

\begin{equation}
EAL_{rank} = \mathrm{rank}(EAL_{composite})
\end{equation}

\begin{equation}
EAL_{score} = \frac{EAL_{composite} - \min(EAL_{composite})}{\max(EAL_{composite}) 
- \min(EAL_{composite})} \times 100\%
\end{equation}

\subsubsection{Community Risk Factor (CRF)}

To generate a Community Risk Factor CRF (CRF) value for a community, its Social Vulnerability
(SV) value is divided by its Community Resilience (CR) value.

\begin{equation}
CRF = f\Big( \frac{SV}{CR} \Big)
\end{equation}

\noindent where $f\Big( \frac{SV}{CR} \Big)$ is a triangular distribution with minimum $0.5$, maximum $2$, and mode $1$.
The selection process for the CRF involved evaluating various shapes, ranges, and modes.

Considering its ability to highlight communities at both ends of the distribution without attributing
extreme values to a select few, we use the triangular distribution. It also takes into account
the $EAL$ as the primary driver of risk, thereby making a mode of 1 the most appropriate choice
for the $CRF$.

The $SV$ values correspond to SOVI values, while the $CR$ values represent HVRI BRIC
index for the community, as provided by the source data sets.

\subsubsection{Risk Calculation}

In the most general terms, natural hazard risk is often defined as the likelihood (or probability)
of a natural hazard event happening multiplied by the expected consequence if a natural
hazard event occurs.

\begin{equation}
Risk = Likelihood \times consequence
\end{equation}

To make the level of risk more concrete and exemplified, we estimate risk from two aspects:
risk value and risk score.

\begin{equation} 
R_{value} = EAL_{value} \times CRF  \label{eq8}
\end{equation} 

\begin{equation} 
R_{score} = EAL_{score} \times CRF  \label{eq9}
\end{equation} 

\noindent where $R$ represents $Risk$.

This risk equation of $R$ includes three components: a natural hazards risk component, a
consequence enhancing component, and a consequence reduction component. $EAL$ is the natural
hazards risk component, measuring the expected loss of building value, population, and/or
agriculture value each year due to natural hazards. $CRF$ incorporates both the consequence
enhancing component, denoted as $SV$, and the consequence reduction component, represented
by $CR$.

For Equation \eqref{eq8}, values for Risk and $EAL$ are in units of dollars, representing the community’s
average economic loss from natural hazards each year. For $SV$ and $CR$, values are the
index values for the community provided by the source data sets.

For Equation \eqref{eq9}, each component is represented by a score that represents a community’s
percentile ranking relative to all other communities at the same level. The composite risk score
is calculated to measure a community’s risk to all $18$ hazard types. And it is also a community’s
percentile ranking in risk compared to all other communities at the same level. The risk score
and $EAL$ score are provided as both composite scores from the summation of all $18$ hazard
types.

We used the K-algorithm to categorize the disaster levels into five classes, as shown in
Figure \ref{fig3}.

%% 并排子图
\begin{figure}[H]    % H表示强制固定在当前位置
\small
\centering
\begin{subfigure}{0.53\textwidth}
\centering
\includegraphics[height=5.5cm]{fig3a.png}
\caption{Risk levels in different regions in U.S.} 
%\label{fig3a}
\end{subfigure}
\begin{subfigure}{0.4\textwidth}
\centering
\includegraphics[height=5.5cm]{fig3b.png}
\caption{Risk level} 
%\label{fig3b}
\end{subfigure}
\caption{Risk rating and map presentation} \label{fig3}
\end{figure}

\subsection{Insurance Premium}

Generally, the companies adopt the principle of Level Premium to determine the price of
insurance. The company calculate the pure premium by using a pre-determined claim rate and
a desired return on investment. The total premium is then obtained by adding surcharges at a
certain expense ratio. Then, the total premium is obtained by adding surcharges at a certain
expense ratio. Insurance companies increase their profitability by increasing expense ratios and
reducing expected returns on investment. In addition, they utilize the ‘Law of Large Numbers’
to set the overall payout ratio of the product so that it basically matches the statistical data, thus
reducing the company's exposure to the risk of fluctuating payout ratios. For catastrophe insurance
pricing, a natural disaster can cause huge property damage when it occurs. The insurance
company may have a risk of becoming insolvent.

In finance, the capital asset pricing model (CAPM) is a model used to determine a theoretically
appropriate required rate of return of an asset, to make decisions about adding assets
to a well-diversified portfolio \cite{li2002}.

The reward-to-risk ratio for any individual security in the market is equal to the market
reward-to-risk ratio, thus

\begin{equation}
\frac{E(r_i)-r_f}{\beta_i} = E(r_m)-r_f
\end{equation}

\begin{equation}
\beta_i = \frac{\mathrm{Cov}(R_i,R_m)}{\mathrm{Var}(R_m)} = \rho(i,m) \frac{\sigma_i}{\sigma_m}
\end{equation}

\noindent where

\begin{itemize}
\item $E(r_{i})$ is the expected return on the capital asset,
\item $E(r_{m})$ is the expected return of the market,
\item $r_{f}$ is the risk-free rate of interest such as interest arising from government bonds,
\item $\beta_{i}$ is the sensitivity of the expected excess asset returns to the expected excess market returns,
\item $\rho(i,m)$ denotes the correlation coefficient between the investment $i$ and the market $m$,
\item $\sigma_{i}$ is the standard deviation for the investment $i$,
\item $\sigma_m$ is the standard deviation for the investment $m$.
\end{itemize}

Expected return on investment (ROI):

\begin{equation}
ROI = \int \frac{-A \times l(p)+(1+x)A \times R_{Value}-I}{I}f(p) \mathrm{d}p
\end{equation}

Variance of ROI:

\begin{equation}
\mathrm{Var}(ROI) = \int \bigg[\frac{-A \times l(p)+A(1+x) \times R_{value}-l-(Ax \times 
R_{value}-l)}{I}\bigg]^2 f(p) \mathrm{d}p
\end{equation}

\noindent where

\begin{itemize}
\item $A$ is the sum insured,
\item $I$ is the invested capital,
\item $l(p)$ is the loss function,
\item $x$ is the surcharge rate.
\end{itemize}

The surcharge rate $x$ is a multiple of the average value of the loss:

\begin{equation}
x = \frac{\Big(1+r_f+\rho \times \frac{(A/I) \times \sigma}{\sigma_M} \times 
(r_M-r_f) \Big) \times I}{A \times R_{value}} \label{eq14}
\end{equation}

where $\sigma$ is the Standard deviation of $R_{value}$, $y$ is Pure premium per $\$10,000$:

\begin{equation}
y = 10000(1+x) \times P
\end{equation}

\noindent where $P$ is the probability of a disaster causing damage.

Typically, it is more reasonable to spend $3-10$ percent of each person's annual income on
insurance.We assume that each person is willing to spend $5\%$ of his or her annual income each
year to purchase catastrophe insurance with a one-year term. Insurance companies can make
decisions from two perspectives based on the above formula:

\begin{itemize}
\item Introducing bankruptcy theory, after calculating the lowest order price, y, in the case
where the probability of the firm's future bankruptcy is less than $10\%$, and then comparing
it to the local per capita annual disposable income (GNI), it is expected that
people in the locality will not be able to afford to consume catastrophe insurance and
will not invest in it if the ratio of premiums per $10,000$ to GNI is greater than $5\%$.

\item We use $5\%$ of the local national GNI per capita as the subscription price per $\$10,000$
of premium. If this price makes the likelihood of future insolvency of the company
higher than $10\%$, no investment is made in that location.
\end{itemize}

The price of insurance also affects people's desire to buy to some extent, and an increase
in the price of insurance may lead to a decrease in their desire to buy.

\begin{equation}
N = (1-\omega y) N_A
\end{equation}

\begin{equation}
Total~Revenue = y \times N
\end{equation}

\noindent where

\begin{itemize}
\item $N_A$ is the total local population,
\item $N$ is the number of local people with a strong desire to buy,
\item $\omega$ is the factor that influences the price of insurance on the willingness of locals to buy, and is related to the average disposable income of locals as well as the gap between the rich and the poor,
\item $Total~Revenue$ is the projected total local insurance revenue.
\end{itemize}

The company first determines the area in which it wants to invest money to build the
insurance and then determines the price of local insurance. We would like to maximize the
company's total revenue:

\begin{equation}
\begin{aligned}
& \max ~~ Total~Revenue \\
& \text{s.t.} ~~ 
\begin{cases}                % cases环境生成左大括号
0 < \omega y < 1 \\
y \geqslant y_{10} \\
y < 0.05 \times GNI
\end{cases}
\end{aligned}
\end{equation}

where $y_{10}$ is the price of insurance when the firm's insolvency rate is $10$ percent.

Our model is implemented in Gorontalo, Indonesia and Los Angeles, California. This is
because both locations have similar and high risk indices, with Los Angeles having the highest
disaster risk index in the United States.

%% 并排放两个子图
\begin{figure}[H]      % H表示强制固定在当前位置
\small
\centering
\begin{subfigure}{0.5\textwidth}
\centering
\includegraphics[height=5.5cm]{fig4a.jpg}
\caption{Gorontalo}    % \label{fig4a}
\end{subfigure}
\begin{subfigure}{0.45\textwidth}
\centering
\includegraphics[height=5.5cm]{fig4b.jpg}
\caption{Los Angeles}   % \label{fig4b}
\end{subfigure}
\caption{Location of the two areas on the map} \label{fig4}
\end{figure}

After searching for relevant data, we calculated that in order to ensure that the probability
of the company's bankruptcy after investing in catastrophe insurance in Gorontalo is less than
$10\%$, we need to charge a premium of $\$342.745$ for every $\$10,000$ of coverage, which is calculated
in equation \eqref{eq14}, of which $\$283.465$ is the pure premium and $\$59.28$ is the additional
premium. Searching for relevant information we find that $5\%$ of the per capita disposable income (GNI) of 
Gorontalo is only $\$137.25$, so the likelihood of residents being willing to purchase
catastrophe insurance is low and the company should not invest in catastrophe insurance
in the area.

In order to ensure that the probability of insolvency of the company after investing in
catastrophe insurance in Los Angeles is less than $10\%$, through the formula \eqref{eq14} 
calculated that for every $\$10,000$ of coverage need to charge a premium of $\$295.09$ of 
which the pure premium is $\$200$ and the additional premium is $\$95.09$ (because of the higher 
return on investment in the market in the U.S.). The per capita disposable income in 
Los Angeles ($5\%$ of GNI is $\$3,162.65$), which is much higher than the cost of catastrophe 
insurance. In order to determine the most appropriate cost of insurance to earn a greater benefit, 
we plotted the trend of total premium income as a function of premiums.

\begin{figure}[H]    % H表示强制固定在当前位置
\small
\centering
\includegraphics[width=0.5\textwidth]{fig5.jpg}
\caption{Relationship between company revenue and insurance price in Los Angeles} \label{fig5}
\end{figure}

From the Figure \ref{fig5}, it can be seen that with the increase of premiums, the total income of
insurance companies tends to increase first and then decrease. This is because when the premium
is too low, although the number of insured people is high, the amount of single transaction
is small and the number of guarantees is too high, which leads to a higher risk of bankruptcy
of the insurance company; whereas too high a premium will reduce the consumer's expectations
of catastrophe insurance, and the volume of insurance orders will be small.

In summary, for Los Angeles, a premium of about $2,500$ per $\$10,000$ of coverage can be
used, and $1.92$ million people are expected to purchase the company's catastrophe insurance
(the total population of Los Angeles is about $3.79$ million). At this point, the insurance company's
theoretical revenue would be around $\$4.5$ billion. Although Los Angeles has a high risk
index, the profits are equally attractive, so the insurance company could take the risk of launching
its catastrophe insurance business here.

\subsection{Measures}

\subsubsection{Insurance Securitization}

Catastrophe bonds are risk-linked securities that transfer a specified set of risks from a
sponsor to investors. Catastrophe bonds emerged from a need by insurance companies to alleviate
some of the risks they would face if a major catastrophe occurred, which would incur
damages that they could not cover by the invested premiums \cite{jaffee1997, froot1999}. An insurance company issues
bonds through an investment bank, which are then sold to investors. These bonds are inherently
risky, and usually have maturities less than 3 years. If no catastrophe occurred, the insurance
company would pay a coupon to the investors. But if a catastrophe did occur, then the principal
would be forgiven and the insurance company would use this money to pay their claim-holders.

\begin{figure}[H]    % H表示强制固定在当前位置
\small
\centering
\includegraphics[width=0.65\textwidth]{fig6.jpg}
\caption{Insurance securitization schema} \label{fig6}
\end{figure}

From an economic perspective, the securitization of insurance, particularly through instruments
like catastrophe bonds, represents a significant innovation in the capital markets.
This innovation not only diversifies investment opportunities but also plays a crucial role in
enhancing the resilience of the insurance industry against catastrophic events. Catastrophe
bonds allow insurance companies to transfer the risk of extreme events, such as natural disasters,
to the capital markets, thereby reducing their potential liability and improving their solvency.
This mechanism enables insurance firms to manage their risk exposure more effectively
and to maintain stability in the face of potentially ruinous events. By doing so, it also ensures
that insurance companies can continue to offer coverage for risks that might otherwise be uninsurable
due to their catastrophic potential.

\subsubsection{Co-operation with the Government}

The government plays an important role in the country. The government can make some
appealing policies to stimulate people to buy insurance and cooperate with insurance companies
to undertake part of the risk. When people buy insurance, individuals are only required to bear part of the premium. The remainder is subsidized by the various levels of government. If
necessary, special groups of people may be fully covered by government finances \cite{michel2011}. When a
catastrophe occurs, the government can act as a reinsurer and bear part of the amount of compensation.
If the amount of compensation is small, the insurance company will pay directly.
Otherwise, it can be covered or partially paid by the government. In this way, a multi-layered
diversification of risk is constructed. It not only brings benefit protection to the people, but also
drives the development of the insurance industry \cite{aase1992}.

\subsubsection{The Implementation of Measures}

Through the two scenarios described above, the insurance company's market return on
investment in the Gorontalo region $r_m$ increased. When a natural disaster occurs, the amount
of compensation paid by the insurance company is shared by the insurance company, the investors
in the insurance securities, and the local government. In addition, the government subsidizes
residents for catastrophe insurance, which increases the willingness of residents to purchase
catastrophe insurance and reduces the actual cost paid by individuals. The insurance
company can set premiums at the lowest premium ($\$295.09$) that can be assumed under the
risk of insolvency. We plot the trend of total premium income as a function of premium at this
point in time.

\begin{figure}[H]    % H表示强制固定在当前位置
\small
\centering
\includegraphics[width=0.5\textwidth]{fig7.jpg}
\caption{The Analog trend of total premium income in Gorontalo} \label{fig7}
\end{figure}

As we can see from the picture. Under the company's affordable insolvency risk, the insurer
expects maximum revenues of $\$245$ million. The company expects maximum revenue is
$\$245$ million. At this time ,each $\$10,000$ of insurance amount charges $\$295.09$ of insurance
premiums. We expect $830,255$ people (about $73.26\%$ of the total population) to have catastrophic
insurance. Gorontalo residents pay only $\$137.25$ individually, and the remainder is
subsidized by the Gorontalo government. The government guarantees the legal rights of Gorontalo
residents as well as their social welfare.

\subsection{Real Estate Decision Making}

Our insurance model has a significant impact on the development decisions of real estate developers. Based on the above model derivation, it can be learned that for areas with high
natural disasters and low per capita income, if the insurance company is willing to underwrite
policies, it will result in the high bankruptcy rate of the company not being able to realize
profitability \cite{blanchard1994}. Similarly, real estate developers will not choose the area for investment and development
due to high risk and lack of demand. That is, any area with high R value and income
(GNI) below a certain value is not recommended for real estate developers to invest in. In
addition to this, such areas have the following risk factors: 

\begin{figure}[H]    % H表示强制固定在当前位置
\small
\centering
\includegraphics[width=0.5\textwidth]{fig8.jpg}
\caption{Risk factors} \label{fig8}
\end{figure}

Further applying our model, we can calculate the insurance rate, which is the insurance
premium divided by insurance amount. If the area has a high insurance rate by calculating,
the property developer would have to bear a higher insurance cost during the construction of
the building as well as during the unsold period. Therefore, property developers need to carefully
consider and weigh the future profit and loss before making decisions.

Similarly, in other areas, we can calculate local insurance rates based on our model. According
to this indicator, property developer can further determine the cost of developing land
in local area and buying insurance. In this way we provide a reference for the property developer’s
decision making.

Additionally, our model can also provide guidance about how property developers build
construction. For each of the $18$ hazard types, we can calculate the value of EAL and Rvalue (in
dollars) for each hazard type. We find a positive correlation between EAL and Rvalue to some
extent. Thus, property developers can determine the different major hazrard type for each area
based on Rvalue and thus build different types of homes. For example, in the city of New Orleans,
USA, flooding ranks high on the list of $18$ natural disasters in terms of Rvalue. Accordingly,
many property companies, such as American Restorators LLC, are building houses with high
foundations locally to minimize damage and achieve business profitability. Our model solves
the problem about how to build on certain site. This approach not only maintains the interests
of real property developers, but also protects the lives of people in the community.

\section{Building Preservation Model}

\subsection{Building Value Quantification}

Building value is measured in terms of the building's cultural value and community influence,
economic value, and historical value. Therefore, we take these three main aspects as primary
indicators.

\subsubsection{Indicators Determination}

For the cultural value and community influence, we synthesized various factors, such as
geography and network, and finally selected the three most representative secondary indicators
to construct our model. Similarly, for the economic and historical value, we selected two secondary
indicators each to improve the model. The specific description and indicators selected
are shown in Table \ref{tbl3}.

% 创建Excel表格(可合并单元格), 导入在线生成表格: https://www.latex-tables.com/
% 设置网格线、右侧可设置背景色

\begin{table}[H]
\small
\centering
\caption{Indicators}  
\label{tbl3}         
\begin{tblr}{
  row{odd} = {LightGreen},
  row{1} = {ForestGreen},
  cell{2}{1} = {r=3}{LightGreen},
  cell{5}{1} = {r=2}{},
  cell{7}{1} = {r=2}{},
  vlines,
  hline{1-2,5,7,9} = {-}{},
  hline{3-4,6,8} = {2-3}{}}
Object    & Indicators & Description      \\
{Cultural Values and \\ Community influence} 
   & NG  & Number of Google search terms   \\
   & P   & {Participation in events held \\ around the building} \\
   & ANV & Annual number of visitors        \\
Economy  & LV         & Land value         \\
         & CC         & Construction cost  \\
History  & NH         & Number of historical research documents  \\
         & DP         & Degree of preservation
\end{tblr}
\end{table}


\begin{itemize}
\item Cultural Value and Community Influence
\begin{description}
\item [$\circ$] Global Visibility

The cultural value of a building depends to a large extent on its global visibility. So we
quantify its global visibility through two metrics, ``Number of Google search terms''
(NG) and ``Annual number of visitors'' (ANV). This approach balances online and offline,
making the measurement of cultural values more quantifiable and accurate.

\item [$\circ$] Impact on the Community

Buildings have a strong connection with local communities. When measuring the value
of a building, we take into account its impact on the local community. Research has
shown that the more influence a building has on the local community, the more the
value of the landmark itself will increase. Besides, it will further promote the increase
of influence, realizing a positive feedback loop. Therefore, we choose ``Participation in
events held around the building'' (P) to quantify the building's influence on the community.
We calculate P as follows:

\begin{equation}
P = \frac{1}{N} \times \sum_{i = 1}^N \frac{NCMP_i}{NTC_i}
\end{equation}

where $NCMP_i$ represents ``Number of community members participating in activities''
at the $i$th activity, $NTC_i$ means the total number of people in the community at the
time of the $i$th activity and N means the total number of activities conducted around
the building.

\end{description}

\item Economy Value

For economic value, we mainly consider the value of the building in terms of its construction.
Therefore, we considered the value of the land it occupies. And it is measured
by the indicator ``Land value'' (LV).

\begin{equation}
LV = P_c \times Area
\end{equation}

where $P_c$ represents the current price of the land and Area represents the area occupied by
the building. Meanwhile, for the value created during the construction of the building itself,
we use ``Construction cost'' (CC) for quantitative assessment. Taking inflation into account,
we define Construction cost as all costs involved in the implementation of that construction
project under this year's Engineering News-Record (ENR) benchmark for the region. Both
the $LV$ and $CC$ metrics are expressed in U.S. dollars.

\item Historic Value

\begin{description}
\item [$\circ$] Historical Research Value

The historical value of a building is largely dependent on its place in historical research.
So we quantify its visibility and importance in the academic world through NH. NH
refers to the number of historical research documents related to the building, including
but not limited to books, papers, reports, etc. This indicator reflects the building's attention
and depth of research in the historical community. The higher NH value means
the building has a higher historical research value.

\item [$\circ$] The Preservation Condition

The historic value of a building is also affected by its state of preservation. We use
``Degree of preservation'' (DP) to measure the extent to which a building has been preserved
from its original state. It includes aspects such as structural integrity, exterior
preservation, and interior decoration. The assessment of DP can be based on expert
review, preservation grade, and comparative analysis with the original state. Highly
preserved buildings not only better transmit history and culture, but also provide rich
materials for future research.
\end{description}
\end{itemize}

\subsubsection{Weight Calculation}

CRITIC is an objective assignment method based on data volatility. The idea of this
method was based on two indicators, contrast intensity and correlation indicators. When calculating
the weights, we need to multiply the contrast intensity with the correlation indicator
and then normalize to get the final weights.

\begin{itemize}
\item Contrast intensity refers to the magnitude of the difference in values between evaluation
programs for the same indicator, expressed as a standard deviation. The larger the standard
deviation, the greater the fluctuation . That is, the larger the difference in the values taken between
the programs, the higher the weight will be.
\item The Sperman correlation coefficient is used to express the correlation between indicators.
If there is a strong positive correlation between two indicators, it means that the less conflicting
they are, the lower the weight will be.
\end{itemize}

1) There are $n$ samples to be evaluated and $p$ evaluation indicators to form the raw indicator
data matrix.
\[
X = \begin{pmatrix}
x_{11} & \cdots & x_{1p} \\
\vdots & \ddots & \vdots \\
x_{n1} & \cdots & x_{np}
\end{pmatrix}
\]

where $x_{ij}$ represents the value of the $j$th evaluation indicator for the $i$th sample.

2) In order to remove the effect of the scale each indicator is normalized. The indicators
we selected are of benefit attributes type, so the normalization formula:
\[
X_{ij} = \frac{x_{ij}-\min(x_j)}{\max(x_j)-\min(x_j)}
\]
\noindent where $X_{ij}$ is normalized to obtain a numerical matrix.

3) Then we calculate the contrast intensity of the indicator:
\[
\left\{
\begin{aligned}
& \bar{x_j} = \frac{1}{n}\sum_{i=1}^{n} x_{ij} \\
& S_j = \sqrt{\frac{\sum_{i=1}^{n}(x_{ij}-\bar{x_j})^2}{n-1}}
\end{aligned}
\right.
\]
\noindent where $S_j$ represents the strength of comparison of the $j$th indicator.

The larger the $S_j$, the greater the difference in values for that indicator. The more information
the indicator reflects, the stronger the evaluation strength of the indicator itself and
the more weight should be assigned to it.

4) Calculation of the conflicting nature of the indicators
\[
\left\{
\begin{aligned}
& d_i = \mathrm{rank}(x_{ij})- \mathrm{rank}(x_{ik}) \\
& r_{jk} = 1-\frac{6 \sum d_i^2}{n (n^2-1)} \\
& R_j = \sum_{k=1,k\neq j}^p (1-r_{jk})
\end{aligned}
\right.
\]
\noindent where $R_{jk}$ denotes the Sperman correlation coefficient between evaluation indicators $j$ and
$k$. $R_j$ denotes the conflictual of the $j$th indicator.

The Sperman correlation coefficient is used to express the correlation between indicators.
The stronger the correlation between two indicators, the less they conflict, the more they reflect
the same information, and the more repetitive the content of the evaluation is. To a certain
extent, the evaluation strength of the indicator is weakened and the weight assigned to it should
be reduced.

5) Calculation of the amount of information:
\[
C_j = S_j \times R_j
\]

6) Based on the amount of information, we calculate the weights of each indicator defined
$w_j$:
\[
w_j = \frac{C_j}{\sum_{j=1}^p C_j}
\]

7) The score for each indicator is:
\[
Score_{object} = \sum_{j=1}^p w_j s_j
\]
\noindent where $object$ represents ``Cultural values and community influence'', ``Economy'', ``History'',
$s_j$ denotes the value of the $j$th secondary indicator.

Applying the CRITIC weighting method for each level 1 indicator separately, the objective
weights for each level 2 indicator were obtained as shown in the following Table \ref{tbl1}.

\subsubsection{Quantitative Results of Building Values}

In order to assign weights to these three level 1 indicators to get the final building value,
we use hierarchical analysis to construct a judgment matrix to get the weights of the three level
1 indicators:
\[
\theta = (0.4432, 0.3873, 0.1694)
\]
\noindent where the consistency ratio of the judgment matrix $= 0.017591$, and the consistency is acceptable.
Ultimately, our building impact score is calculated as follows:

\begin{equation}
V_{score} = \sum_{i=1}^3 \theta_i Score_{object}
\end{equation}

% 创建Excel表格(可合并单元格), 导入在线生成表格: https://www.latex-tables.com/
% 设置网格线、右侧可设置背景色

\begin{table}[H]
\small
\centering
\caption{The weight of indicators}
\label{tbl4}
\begin{tblr}{
  cells = {LightGreen},
  row{1} = {ForestGreen},
  row{3} = {},
  cell{2}{1} = {r=3}{},
  cell{2}{2} = {r=3}{Wheat},
  cell{5}{1} = {r=2}{},
  cell{5}{2} = {r=2}{Wheat},
  cell{5}{3} = {},
  cell{5}{4} = {},
  cell{7}{1} = {r=2}{},
  cell{7}{2} = {r=2}{Wheat},
  cell{7}{3} = {},
  cell{7}{4} = {},
  vlines,
  hline{1-2,5,7,9} = {-}{},
  hline{3-4,6,8} = {3-4}{}}
Object  & Weight & Indicators & Weight \\
{Cultural Values and \\ Community influence} 
   & 0.4432 & NG  & 0.1698 \\
   &        & P   & 0.4429 \\
   &        & ANV & 0.3873 \\
Economy  & 0.3873 & LV         & 0.4272 \\
         &        & CC         & 0.5728 \\
History  & 0.1694 & NH         & 0.6286 \\
         &        & DP         & 0.3714 
\end{tblr}
\end{table}

% 更多颜色https://ctan.math.washington.edu/tex-archive/macros/latex/contrib/xcolor/xcolor.pdf

\subsection{Determination of protection measures}

\subsubsection{Measure Score}

In Model 1, we obtained a composite risk score $R_{score}$ of each region by analyzing $18$
natural hazards. Next, in the above section, we quantified the value of the building to get the
score $V_{score}$. By multiplying the risk score and the value score, $M_{score}$ is obtained, which is
used to assess the conservation priority of the building and the extent and scale of conservation
measures that need to be taken.

\begin{equation}
M_{score} = V_{score} \times R_{score}
\end{equation}

A higher $M_{score}$ indicates a higher value of the building, along with a higher risk of exposure
to natural hazards. Therefore, more urgent and comprehensive protection measures are
needed. Based on the statistical distribution of $M_{score}$, we set reasonable thresholds to recognize
low, medium, and heigh grades, and the values of the specific thresholds need to be set
based on expert recommendations and industry standards.

\subsubsection{Score of Protection Measures}

\begin{itemize}
\item {\bf Low:} For low-grade $M_{score}$ buildings, basic conservation measures, such as routine
maintenance and inspections and, where necessary, minor repairs, are undertaken. The risk or
value of these buildings is low, so the measures taken are mainly preventive and low-cost.
\item {\bf Medium:} For medium-grade $M_{score}$, moderate protection measures are implemented, including
enhanced structural inspections, improved safety features, and disaster preparedness
programs. These measures aim to increase the resistance and resilience of buildings and require
moderate investment.
\item {\bf High:} For high-grade $M_{score}$, implement comprehensive and high-intensity protection
measures. This may include comprehensive structural reinforcement, installation of advanced
security systems, and customized risk management plans in cooperation with external experts.
Given the high risk or value of these buildings, the goal of the measures is to minimize potential
losses, even if this means higher initial costs.
\end{itemize}

{\bf Note that:} The ``one-size-fits-all'' approach to disaster protection measures ignores the impact
of regional differences, architectural characteristics and socio-economic factors, and can
lead to poor protection and resource utilization. So specific protection measures still need to be
derived from a thoughtful local analysis by natural disaster experts.

\subsubsection{Mentoring for Community Leaders}

Our model provides a quantitative and systematic framework for community leaders to
help them determine the extent and priority of preservation measures based on a building's risk

and value scores. The model makes the decision-making process more scientific and accurate
by combining risk scores (which consider threats such as natural hazards) and value scores
(which include historical, cultural, economic, and community importance). It promotes optimal
allocation of resources and ensures that high-value or high-risk buildings are adequately protected,
while also taking into account economic benefits. In addition, the model encourages
community involvement and support, improves disaster response capacity, and supports sustainable
community development. Through this approach, community leaders are able to make
more informed decisions to protect and maintain important buildings in their neighborhoods,
contributing to the overall well-being and development of the community.

\section{Landmark Case Analysis}

We select Tokyo Tower in Japan as the landmark for evaluation and analysis.

\begin{figure}[H]    % H表示强制固定在当前位置
\small
\centering
\includegraphics[width=0.5\textwidth]{fig9.jpg}
\caption{Location of Tokyo Tower} \label{fig9}
\end{figure}

\subsection{Insurance Pricing for Tokyo Tower}

Designed by Japanese architect Tachu Naito, the Tokyo Tower cost $\$8.4$ million to build
at the time and was constructed to send broadcast signals in Tokyo. Currently, Tokyo's GNI
per capita is $\$36,964.96$, and the Tokyo capital market is currently functioning well. The risk
score for Tokyo is calculated to be 86. According to the insurance pricing model, the optimal
insurance rate for an insurance company to issue catastrophe insurance in Tokyo is $3.8\%$. In
view of the special historical value of the Tokyo Tower, the insurance company may appropriately
increase the insurance rate. As a result, the Tokyo government spends approximately US
$\$319,200$ per year on catastrophe insurance for the Tokyo Tower.

\subsection{Architectural Value of Tokyo Tower}

In 2011, a major earthquake struck northeastern Japan, and this earthquake caused some
damage to Tokyo Tower, bending the antenna at the tip of the tower by $2$ degrees, resulting in
the end of the experimental broadcasting of terrestrial wave digital sound broadcasting, and the
interruption of the transmission of the $24/7$ terrestrial analog television signals. 
Tokyo Tower entered a maintenance period during which analog signals and FM broadcasting-related services
also began to be transferred to Tokyo Skytree, so the proportion of Tokyo Tower's actual
use gradually decreased. However, Tokyo Tower has attracted more than $3$ million visitors as
a tourist attraction and has accumulated more than $150$ million visitors. It is also a symbol of
Japan's post-war prosperity and has a remarkable historical significance for the Japanese people.

According to the calculations of the building conservation model, the Mscore of Tokyo
Tower is located in a high conservation level area. Therefore, the government should consider
purchasing catastrophe insurance for Tokyo Tower and strengthening its daily supervision and
maintenance to ensure that it can withstand natural disasters, such as earthquakes, and that its
seismic treatment measures also need to be strengthened.

\section{Sensitivity and Robustness Analysis}

\subsection{Sensitivity}

In section 4.3, factor is introduced to estimate the parameters of the expected return
of the market. Therefore, change the size of this parameter, that is, the capital market environment
has changed. Below we analyze the sensitivity of this parameter. Gradual reduction of
the parameter by $5\%$. The reason for considering only a decrease in r and not an increase in r
is to reflect the worst-case scenario, i.e. a gradual decrease in capital market returns, and to see
if our model is sensitive to the parameter.

Therefore, Re-simulate the calculation results and obtain $3$ sets of curves as shown in
Figure \ref{fig10}.

% 并排两个图(非子图)
\begin{figure}[H]
\begin{floatrow}
\ffigbox[\FBwidth]
{
\includegraphics[width=0.48\textwidth]{fig10.jpg}
}
{\caption{Sensitivity analysis of $r_m$} \label{fig10}}
\ffigbox[\FBwidth]
{
\includegraphics[width=0.48\textwidth]{fig11.jpg}
}
{\caption{Robustness analysis of $R_{value}$} \label{fig11}}
\end{floatrow}
\end{figure}

The results show that as r decreases, the profit gained by the insurance company if it sells
the insurance company at the same price tends to decrease. This makes sense because lower
capital market interest rates result in lower profits for insurance companies. The trend of the
curve obtained by sensitivity test is consistent with the actual situation.

\subsection{Robustness}

We verify the robustness of the model. Given the uncertainty of natural hazards, we may
have errors in the calculation of Rvalue, which affects our grading of the risk indicators for each
region. Randomly selecting some of the more than $3,000$ counties in the United States and
deviating its Rvalue by $5\%$, and again grading these areas, the results obtained are shown in
Figure \ref{fig11}.

As can be seen from the figure, the grading of the original regions changes only slightly
after randomly selecting regions to bias their measurements. This indicates that the small error
in $R$ does not cause large changes in the model results and our model is stable.


\section{Model Evaluation}

\subsection{Strengths}

{\bf Robustness and Flexibility:} Our model demonstrates strong adaptability to various parameter
variations through sensitivity analysis and robustness testing, and is able to provide
reliable predictions under different scenarios.

{\bf Comprehensive consideration:} Our model integrates social and economic factors, natural
disaster risks, and provides a way to price insurance.

\subsection{Weaknesses}

{\bf Data Dependency:} Although the model has a high demand for data quality, it reflects our
scientific attitude of pursuing accuracy and real-time performance. By working with data providers,
we can continuously optimize the data collection and processing process.

\printbibliography

\newpage

\begin{center}
{\Large \textbf{A Letter to Community Members}}   
\end{center}

\vspace{0.5cm}

\noindent Dear Esteemed Members of Tokyo Tower Community,

We are researchers dedicated to the study of architectural preservation. In order to realize the preservation of Tokyo Tower at a lower cost, please allow me to introduce our proposal on behalf of our team.

We have recently concluded a comprehensive analysis aimed at ensuring the future sustainability and preservation of our cherished Tokyo Tower. Our findings offer a strategic blueprint that prioritizes not only the physical well-being of this iconic structure but also its cultural and historical essence.


\textbf{Recommendation Plan:}

Our proposed plan encompasses innovative preservation techniques, structural enhancements, and community-centric initiatives designed to safeguard and celebrate the tower's legacy. We aim to implement cutting-edge solutions that address both current vulnerabilities and future challenges.


\textbf{Implementation Timeline:}

This ambitious project is scheduled over a five-year period, beginning with immediate preliminary assessments and followed by successive phases of structural upgrades and community engagement activities. This phased approach ensures meticulous attention to detail and the successful realization of our objectives.


\textbf{Budget Overview:}

The financial blueprint for this venture is estimated at $\$320,000$ per annum. This budget encompasses costs associated with structural reinforcements, preservation technology, and community enrichment programs. It represents a balanced investment in tower's longevity and its significance to our community.

We stand at a pivotal moment in the history of Tokyo Tower, poised to embark on a journey that not only secures its future but also reinvigorates its role within our community. We invite you to support this noble cause, ensuring that Tokyo Tower remains a symbol of resilience, heritage, and communal unity for countless years to come.

\vspace{1.5cm}                    % 增加间距,把落款往下推

\hfill Sincerely yours,           % 右对齐

\hfill {\bf Team \#2615744}       % 右对齐

\newpage

\begin{appendices}

\section{First Appendix}

\textbf{\textcolor[rgb]{0.98,0.00,0.00}{Input Python source:}}
\begin{lstlisting}[language=python]
import numpy as np

def fft(x):
    n = len(x)
    if n == 1:
        return x
    even = fft(x[0::2])
    odd = fft(x[1::2])
    T = [np.exp(-2j * np.pi * k / n) * odd[k] for k in range(n // 2)]
    return [even[k] + T[k] for k in range(n // 2)] + [even[k] - T[k] for k in range(n // 2)]
\end{lstlisting}

\textbf{\textcolor[rgb]{0.98,0.00,0.00}{Input Matlab source:}}
% 兼容两种编译方式:
% 1) 在 MCM_Latex2026 目录内编译:code/mcmthesis-matlab1.m
% 2) 在项目根目录编译:MCM_Latex2026/code/mcmthesis-matlab1.m
\IfFileExists{code/mcmthesis-matlab1.m}{%
\lstinputlisting[language=Matlab]{code/mcmthesis-matlab1.m}%
}{%
\lstinputlisting[language=Matlab]{MCM_Latex2026/code/mcmthesis-matlab1.m}%
}

\end{appendices}


\AImatter

\begin{ReportAiUse}{9}

\bibitem{AI1}
OpenAI ChatGPT \cite{chatgpt2024} (Jan 25, 2024 version, ChatGPT-4) 

Query1: Give me some brainstorming ideas for innovative data visualization techniques.

Output: Here are some innovative data visualization technique ideas:

(1) Interactive Dashboards: Develop dashboards that allow users to interact with the data,
such as filtering, zooming, and selecting specific data points to see detailed information.

(2) Animated Visualizations: Use animations to show changes in data over time, helping
to illustrate trends, cycles, or progressions.

(3) Augmented Reality (AR) Visualizations: Implement AR to overlay data on real-world
objects or environments, enhancing understanding of spatial and contextual relationships.

(4) 3D Visualizations and Models: Employ 3D models to represent complex data structures,
allowing exploration from different angles and perspectives.

(5) Narrative Visualizations: Combine data visualization with storytelling, guiding viewers
through a narrative that explains the significance of the data in a compelling way.

(6) Real-time Data Streaming: Visualize data in real-time, updating visuals as new data
comes in, ideal for monitoring and tracking systems.

(7) Data Sonification: Convert data points into audio signals, making it possible to ``listen''
to the data for patterns, trends, or anomalies.

These ideas can help present data in more engaging, insightful, and accessible ways.

\bibitem{AI2}
OpenAI ChatGPT (Jan 25, 2024 version, ChatGPT-4)

Query1: What is Engineering News-Record? Give me a brief introduction.

Output: Engineering News-Record (ENR) is a prominent publication within the engineering and
construction industry, providing news, analysis, commentary, and data vital for professionals
in the field. The publication's roots can be traced back to two separate publications from the
late 19th century, eventually merging into what we know today as ENR in 1917. It has evolved
to offer a wide range of content including top design firms, construction companies, projects
in the U.S. and around the world, and industry trends affecting business management, design,
construction methods, technology, and more. ENR is recognized for its comprehensive rankings
of the largest construction and engineering firms, based on gross revenues, and for its ENR
Award of Excellence, which honors individuals who have made significant contributions to the
industry. Today, ENR is owned by BNP Media and continues to serve as an essential resource
for contractors, project owners, engineers, architects, and other industry stakeholders with its
print and digital magazine, website, newsletters, webinars, and events.

\bibitem{AI3}
OpenAI ChatGPT (Jan 25, 2024 version, ChatGPT-4)

Query1: What's the ENR for?

Output: The Engineering News-Record (ENR) serves several vital functions in the engineering and construction industry. It is recognized for:

(1) News and Analysis: ENR provides comprehensive news coverage, analysis, and commentary
on various topics relevant to the construction industry, including major projects, industry
trends, technological advancements, and issues affecting the sector such as safety, law,
and labor.

(2) Rankings and Data: It publishes annual rankings of the largest construction and engineering
firms based on gross revenues, including lists like the Top 400 U.S. General Contractors,
Top 500 U.S. Design Firms, and Top 600 Specialty Contractors, among others. These
rankings are widely respected and used as benchmarks within the industry.

(3) Awards and Recognition: ENR recognizes industry excellence through various awards,
including the ENR Award of Excellence, which honors individuals and projects that have made
significant contributions to the industry.

(4) Industry Insights: The publication offers insights on business management, project
delivery, technology, workforce development, and other critical areas impacting the engineering
and construction fields.

ENR serves as a key resource for professionals across the construction industry, including
contractors, project owners, engineers, architects, and suppliers, connecting diverse sectors
with coverage that addresses common challenges and opportunities.

\end{ReportAiUse}

\end{document}

%% This work consists of these files mcmthesis.dtx,
%%                                   figures/ and
%%                                   code/,
%% and the derived files             mcmthesis.cls,
%%                                   mcmthesis-demo.tex,
%%                                   README,
%%                                   LICENSE,
%%                                   mcmthesis.pdf and
%%                                   mcmthesis-demo.pdf.
%%
%% End of file `mcmthesis-demo.tex'.
